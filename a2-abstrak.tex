%==================================================================
% Ini adalah abstrak dalam bahasa indonesia 
%==================================================================

%% DILARANG EDIT BAGIAN INI
\clearpage
\phantomsection
\addcontentsline{toc}{chapter}{ABSTRAK}
\begin{center}
    \textbf{\large{\judulid}}\\[0.5cm]
    Oleh\\
    \penulis\\
    NIM: \nim\\[2em]
    \textbf{ABSTRAK}\\[0.5cm]
\end{center}
%%----------------------------------------------------------------

%% edit bagian ini
Abstrak adalah sebuah ringkasan singkat yang menjelaskan secara umum tentang isi dari laporan tugas akhir. Abstrak ditulis dalam tiga (3) paragraf yang berisi beberapa kalimat yang menyatakan tujuan, metode, hasil, dan kesimpulan dari laporan tugas akhir. Paragraf pertama berisi latar belakang dan tujuan tugas akhir. Paragraf kedua berisi metode dan pembahasannya. Paragraf ketiga berisi hasil dan simpulan dari tugas akhir yang dikerjakan.

Abstrak harus menjelaskan secara jelas dan singkat apa yang dibahas dalam laporan tugas akhir, mengapa penelitian ini penting dan apa yang ditemukan dari penelitian tersebut. Abstrak harus ditulis dengan bahasa yang mudah dipahami dan harus mencakup informasi penting yang dibahas dalam laporan tugas akhir. 

Abstrak harus mengandung kata-kata yang relevan dengan laporan tugas akhir dan ditulis dengan bahasa yang formal dan akademik. Abstrak merupakan bagian penting dari sebuah laporan tugas akhir karena merupakan bagian yang pertama kali dibaca oleh pembaca dan harus dapat memberikan gambaran yang jelas tentang isi dari laporan tugas akhir. Oleh karena itu, abstrak harus ditulis dengan baik dan sebaik mungkin agar dapat memberikan gambaran yang jelas tentang laporan tugas akhir yang ditulis. Panjang abstrak sebaiknya dicukupkan dalam satu halaman, termasuk kata kunci. Tiga kata kunci dipandang cukup, yang masing-masingnya memuat paduan kata utama, yang dapat merepresentasikan isi Abstrak.\\[0.6cm]
%% edit sampai sini

%% DILARANG EDIT BAGIAN INI
\noindent Kata kunci: \katakunci
%%----------------------------------------------------------------