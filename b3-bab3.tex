%==================================================================
% Ini adalah bab 3
% Silahkan edit sesuai kebutuhan, baik menambah atau mengurangi \section, \subsection
%==================================================================

\chapter[KONSEP RANCANGAN ALAT DAN PENGUJIAN]{\\ KONSEP RANCANGAN ALAT DAN PENGUJIAN}

\section{Metode Pengerjaan Project Berbasis Engineering Design Process}
Bagian ini menjelaskan metode pengerjaan proyek yang mengadopsi pendekatan Engineering Design Process. Pendekatan ini digunakan untuk memastikan perancangan dan pengembangan proyek dilakukan secara sistematis, mulai dari identifikasi masalah hingga evaluasi akhir. Langkah-langkah dalam metode ini membantu dalam mencapai solusi yang optimal dan terukur.

\subsection{Identifikasi Masalah}
Tahap identifikasi masalah bertujuan untuk menguraikan permasalahan utama yang dihadapi dan memerlukan solusi. Pada tahap ini, dilakukan analisis untuk memahami aspek-aspek penting dari masalah dan menentukan faktor-faktor yang perlu diatasi melalui proyek ini.

\subsection{Definisi Kebutuhan}
Berdasarkan masalah yang telah diidentifikasi, tahap ini berfokus pada definisi kebutuhan proyek secara jelas dan terstruktur. Kebutuhan ini meliputi spesifikasi teknis, fungsi yang diinginkan, serta kriteria-kriteria lain yang harus dipenuhi agar solusi dapat berfungsi dengan baik.

\subsection{Generasi Ide dan Solusi}
Pada tahap ini, berbagai ide dan solusi alternatif dikembangkan dan dievaluasi. Bagian ini menjelaskan proses brainstorming untuk menghasilkan ide yang inovatif, termasuk analisis terhadap kelebihan dan kekurangan dari setiap alternatif solusi yang diusulkan.

\subsection{Perencanaan dan Desain Awal}
Desain awal sistem dikembangkan berdasarkan solusi yang dipilih, dengan mempertimbangkan aspek teknis dan kebutuhan yang telah didefinisikan. Bagian ini menyajikan perencanaan mengenai struktur sistem, alat, dan bahan yang akan digunakan, serta jadwal kerja proyek secara keseluruhan.

\subsection{Pembuatan Prototipe}
Prototipe dibuat untuk menguji konsep dan desain awal dari sistem. Bagian ini menguraikan langkah-langkah dalam proses pembuatan prototipe, termasuk alat dan bahan yang diperlukan, serta tantangan yang mungkin dihadapi selama proses.

\subsection{Pengujian dan Evaluasi}
Prototipe yang telah dibuat kemudian diuji untuk menilai kinerjanya terhadap kebutuhan dan spesifikasi yang telah ditetapkan. Bagian ini menjelaskan prosedur pengujian yang diterapkan, metode pengumpulan data, serta analisis terhadap hasil yang diperoleh.

\subsection{Perbaikan dan Penyempurnaan}
Berdasarkan hasil pengujian dan evaluasi, dilakukan perbaikan untuk menyempurnakan sistem. Bagian ini menjelaskan penyesuaian yang dilakukan untuk meningkatkan kinerja sistem, serta proses iterasi yang dilakukan hingga mencapai hasil yang optimal.

\section{Perancangan Sistem Elektronika}
Bagian ini menguraikan perancangan dari sisi elektronika yang menjadi inti dari sistem. Perancangan ini mencakup blok diagram, pemilihan komponen, dan perancangan rangkaian.

\subsection{Blok Diagram Sistem}
Blok diagram sistem memberikan gambaran umum mengenai arsitektur sistem secara keseluruhan. Bagian ini menyajikan diagram beserta penjelasan fungsi setiap blok yang terdapat di dalam sistem, termasuk bagaimana setiap blok berinteraksi.

\subsection{Pemilihan dan Spesifikasi Komponen Elektronika}
Pemilihan komponen elektronika dilakukan berdasarkan kebutuhan dari desain sistem. Bagian ini menjelaskan spesifikasi teknis dari setiap komponen yang digunakan, seperti mikrokontroler, sensor, aktuator, dan komponen pendukung lainnya.

\subsection{Perancangan Rangkaian Elektronika}
Perancangan rangkaian elektronika bertujuan untuk mencapai fungsionalitas yang diinginkan dari sistem. Bagian ini menguraikan skema rangkaian, penjelasan aliran arus dan tegangan, serta hubungan antar komponen dalam sistem.

\section{Perancangan Mekanik}
Bagian ini membahas perancangan mekanik dari sistem yang mendukung komponen elektronik secara fisik, termasuk struktur dan pemilihan bahan.

\subsection{Spesifikasi Desain Mekanik}
Desain mekanik disusun berdasarkan kebutuhan fisik sistem untuk memastikan bahwa komponen elektronika terlindungi dan dapat berfungsi dengan baik. Bagian ini menjelaskan spesifikasi teknis desain mekanik yang digunakan.

\subsection{Pemilihan Bahan dan Komponen Mekanik}
Pemilihan bahan didasarkan pada kriteria seperti kekuatan, ketahanan, dan biaya. Bagian ini menguraikan bahan dan komponen mekanik yang digunakan untuk konstruksi sistem, serta alasan pemilihan bahan tersebut.

\subsection{Desain Struktur dan Konstruksi}
Struktur dan konstruksi sistem dirancang untuk memberikan dukungan fisik yang stabil. Bagian ini menjelaskan proses desain dan konstruksi dari struktur mekanik sistem, serta tantangan yang mungkin dihadapi.

\section{Perancangan Perangkat Lunak}
Bagian ini mencakup perancangan perangkat lunak yang mengendalikan sistem, meliputi diagram alir, pengembangan kode, dan pengujian perangkat lunak.

\subsection{Flowchart atau Diagram Alir Perangkat Lunak}
Diagram alir menggambarkan alur kerja dari perangkat lunak yang mengontrol sistem. Bagian ini menyajikan flowchart lengkap yang menunjukkan logika dan struktur kontrol dari perangkat lunak.

\subsection{Pemrograman dan Pengembangan Kode}
Kode perangkat lunak dikembangkan untuk mendukung fungsionalitas sistem. Bagian ini menguraikan struktur dan logika dari program yang dibuat, bahasa pemrograman yang digunakan, serta strategi pengembangan kode.

\subsection{Pengujian Kode Perangkat Lunak}
Setelah kode perangkat lunak dikembangkan, dilakukan pengujian untuk memastikan bahwa perangkat lunak berfungsi sesuai yang diharapkan. Bagian ini menjelaskan metode pengujian, jenis uji (misalnya, uji unit dan uji integrasi), serta hasil yang diperoleh.

\section{Perancangan Integrasi Sistem}
Bagian ini menjelaskan proses integrasi antara komponen elektronik, mekanik, dan perangkat lunak agar sistem dapat bekerja sebagai satu kesatuan.

\subsection{Integrasi Komponen Elektronika, Mekanik, dan Perangkat Lunak}
Proses integrasi bertujuan untuk menyatukan seluruh komponen sistem agar berfungsi sebagai satu kesatuan. Bagian ini menguraikan langkah-langkah dalam proses integrasi dan memastikan semua komponen bekerja secara sinkron.

\subsection{Pengujian Awal dan Penyempurnaan Integrasi}
Pengujian awal dilakukan setelah integrasi untuk menilai performa sistem secara keseluruhan. Bagian ini menjelaskan hasil pengujian integrasi dan modifikasi yang diperlukan untuk meningkatkan kinerja sistem.

\section{Rencana Pengujian}
Bagian ini merencanakan pengujian yang komprehensif terhadap sistem untuk memastikan bahwa semua komponen dan fungsi bekerja dengan baik.

\subsection{Metode Pengujian Sistem}
Metode pengujian dipilih berdasarkan tujuan dan kebutuhan proyek. Bagian ini menjelaskan berbagai metodologi pengujian yang dirancang, seperti uji kinerja, uji stabilitas, dan uji kompatibilitas.

\subsection{Prosedur Pengujian}
Prosedur pengujian disusun untuk menguji sistem secara menyeluruh, mulai dari pengaturan awal hingga pelaksanaan uji. Bagian ini menjelaskan langkah-langkah pengujian secara detail untuk menjamin konsistensi dan keandalan hasil.

\subsection{Kriteria Keberhasilan Pengujian}
Kriteria keberhasilan ditentukan untuk mengevaluasi kinerja sistem berdasarkan parameter-parameter tertentu. Bagian ini menguraikan kriteria-kriteria keberhasilan yang digunakan, seperti ketepatan, keandalan, dan efisiensi sistem.

Pada bagian ini akan dijelaskan beberapa tahap utama dalam proses pengembangan proyek dan beberapa rekomendasi peningkatan yang dapat dilakukan. Daftar tahapan pengembangan proyek akan disajikan dalam bentuk bernomor untuk menunjukkan urutan logis dari proses, sementara rekomendasi peningkatan akan disajikan dalam bentuk daftar berpoin untuk mempermudah identifikasi setiap item secara mandiri.

\section{Penulisan dengan \LaTeX}
Bagian ini menyediakan tutorial singkat mengenai beberapa lingkungan penulisan dasar yang sering digunakan dalam dokumen LaTeX. Penjelasan ini bertujuan untuk mempermudah pengguna dalam menulis dan mengatur dokumen dengan lebih efisien. Setiap bagian akan disertai contoh kode dan hasilnya untuk membantu pemahaman.

\subsection{List atau Daftar dengan \texttt{packed\_enum}}
Lingkungan \texttt{packed\_enum} digunakan untuk membuat daftar bernomor dengan jarak yang lebih rapat antar item. Ini sangat berguna untuk menampilkan langkah atau tahapan yang memiliki urutan. Berikut adalah contoh penggunaannya:

\begin{lstlisting}
    \begin{packed_enum}
        \item Langkah pertama adalah mengidentifikasi masalah yang ingin diselesaikan.
        \item Langkah kedua melibatkan analisis kebutuhan.
        \item Langkah ketiga adalah mengembangkan ide dan solusi alternatif.
        \item Langkah keempat adalah melakukan pengujian awal untuk mengevaluasi performa.
    \end{packed_enum}
\end{lstlisting}
    
Hasilnya akan tampak seperti berikut:
\begin{packed_enum}
    \item Langkah pertama adalah mengidentifikasi masalah yang ingin diselesaikan.
    \item Langkah kedua melibatkan analisis kebutuhan.
    \item Langkah ketiga adalah mengembangkan ide dan solusi alternatif.
    \item Langkah keempat adalah melakukan pengujian awal untuk mengevaluasi performa.
\end{packed_enum}

\subsection{List atau Daftar dengan \texttt{packed\_item}}
Lingkungan \texttt{packed\_item} digunakan untuk membuat daftar berpoin dengan jarak antar item yang lebih rapat, cocok untuk poin-poin yang tidak memerlukan urutan tertentu. Berikut adalah contoh penggunaannya:

\begin{lstlisting}
    \begin{packed_item}
        \item Meningkatkan kualitas sensor untuk akurasi yang lebih baik.
        \item Menambahkan modul komunikasi untuk kontrol jarak jauh.
        \item Mengoptimalkan kode untuk efisiensi.
        \item Menambah fitur penghematan energi.
    \end{packed_item}
\end{lstlisting}

Hasilnya akan tampak seperti berikut:
\begin{packed_item}
    \item Meningkatkan kualitas sensor untuk akurasi yang lebih baik.
    \item Menambahkan modul komunikasi untuk kontrol jarak jauh.
    \item Mengoptimalkan kode untuk efisiensi.
    \item Menambah fitur penghematan energi.
\end{packed_item}

\subsection{Menuliskan Kode dengan Listing}
Lingkungan \texttt{lstlisting} memungkinkan kita untuk menuliskan atau menyisipkan kode Python, C++, Arduino, Java atau lainnya dalam dokumen LaTeX dengan format yang rapi dan terstruktur. Pada bagian ini, kita akan melihat dua cara untuk menuliskan kode Python: secara langsung di dalam dokumen dan dengan mengambil dari file eksternal.

\Cref{lst:python_direct} menunjukkan fungsi Python yang menghitung faktorial dari sebuah angka. Kode ini ditulis langsung di dalam dokumen LaTeX menggunakan lingkungan \texttt{lstlisting} dengan format diawali dengan \texttt{\textbackslash begin\{lstlisting\}[language=Python, caption=Contoh Kode Python Langsung, label=lst:python\_direct]} dan diakhiri dengan \texttt{\textbackslash end\{lstlisting\}}, dimana:
\begin{packed_item}
    \item \texttt{language=Python}: Mengatur pewarnaan sintaksis untuk Python.
    \item \texttt{caption}: Menambahkan keterangan di atas kode untuk menjelaskan isi kode.
    \item \texttt{label}: Menambahkan label untuk memudahkan referensi kode dalam dokumen.
\end{packed_item}

\begin{lstlisting}[language=Python, caption=Contoh Kode Python Langsung, label=lst:python_direct]
    def factorial(n):
        if n == 0:
            return 1
        else:
            return n * factorial(n-1)
\end{lstlisting}

Jika Anda memiliki file kode Python di folder tertentu (misalnya, di \texttt{kode/code\_sample.py}), Anda bisa menyisipkan kode tersebut langsung ke dalam dokumen LaTeX menggunakan perintah \texttt{\textbackslash lstinputlisting}. Berikut \cref{lst:python_file} dengan format penulisan \texttt{\textbackslash lstinputlisting[language=Python, caption=Contoh Kode Python dari File, label=lst:python\_file]\{kode/code\_sample.py\}}, dimana:
\begin{packed_item}
    \item \texttt{language=Python}: Mengatur pewarnaan sintaksis untuk Python.
    \item \texttt{caption}: Menambahkan keterangan untuk kode yang diambil dari file.
    \item \texttt{label}: Menambahkan label untuk referensi.
    \item \texttt{\{kode/code\_sample.py\}}: Menentukan path atau lokasi file Python yang akan disisipkan. Pastikan file berada di dalam folder \texttt{kode} atau path yang sesuai.
\end{packed_item}

\lstinputlisting[language=Python, caption=Contoh Kode Python dari File, label=lst:python_file]{kode/code_sample.py}

\subsection{Menambahkan Gambar}
Untuk menambahkan gambar hal yang harus dilakukan adalah:
\begin{packed_enum}
    \item Menyalin file gambar (dalam format jpg \/ png) ke dalam folder \textit{gambar}
    \item Mengganti nama file dari gambar agar mudah dikenali, jangan diberi nama gambar-1,-2, dst
    \item Memasukkan seperti \cref{lst:kode_gambar}
\end{packed_enum}

\begin{lstlisting}[language=TeX, caption=Kode untuk Menyisipkan Gambar dalam Dokumen, label=lst:kode_gambar]
    \begin{figure}[H]
        \centering
        \includegraphics[scale=0.2]{gambar-kucing.jpg}
        \caption{Gambar Kucing Lucu dan Imut}
        \label{fig:kucing}
    \end{figure}
\end{lstlisting}

\noindent Berikut adalah penjelasan dari setiap baris pada kode di atas:

\begin{packed_enum}
    \item \texttt{\textbackslash begin\{figure\}[H] ... \textbackslash end\{figure\}}: Membuat lingkungan \texttt{figure} untuk menyisipkan gambar. Parameter \texttt{[H]} digunakan agar gambar diletakkan tepat di posisi yang ditentukan dalam kode. Opsi \textit{H} dapat diganti dengan \textit{h, t, b, p} sesuai kebutuhan.
    \item \texttt{\textbackslash centering}: Mengatur gambar agar berada di tengah halaman.
    \item \texttt{\textbackslash includegraphics[scale=0.2]\{gambar-kucing.jpg\}}: Memasukkan gambar dengan nama file \texttt{gambar-kucing.jpg}. Parameter \texttt{scale=0.2} mengatur ukuran gambar pada 20\% dari ukuran aslinya. Ubah nilainya untuk memperbesar atau memperkecil gambar.
    \item \texttt{\textbackslash caption\{Gambar Kucing Lucu dan Imut\}}: Menambahkan keterangan (caption) di bawah gambar yang akan muncul di Daftar Gambar dan disertai nomor gambar secara otomatis.
    \item \texttt{\textbackslash label\{fig:kucing\}}: Memberikan label pada gambar untuk merujuk gambar ini dalam teks menggunakan \texttt{\textbackslash cref\{fig:kucing\}} atau \texttt{\textbackslash ref\{fig:kucing\}} yang menghasilkan "Gambar 1" atau penomoran gambar sesuai urutan.
\end{packed_enum}

Hasilnya adalah terlihat seperti pada \cref{fig:kucing}.

\begin{figure}[H]
    \centering
    \includegraphics[scale=0.1]{gambar-kucing}
    \caption{Gambar Kucing Lucu dan Imut dengan scala 0.1}
    \label{fig:kucing}
\end{figure}

Setiap gambar harus dimention atau disebutkan didalam bacaan seperti berikut ini \cref{fig:kucing} dan \cref{fig:logoUNY}.

\begin{figure}[H]
    \centering
    \includegraphics[scale=0.4]{logo-uny}
    \caption{Logo UNY dengan scala 0.4}
    \label{fig:logoUNY}
\end{figure}

Untuk menyisipkan beberapa gambar dalam satu kelompok dan satu caption utama, kita dapat menggunakan lingkungan \texttt{subfigure} di dalam lingkungan \texttt{figure}. Metode ini sangat bermanfaat jika kita ingin menyusun beberapa gambar berukuran kecil dalam satu baris atau kolom, dengan setiap gambar diberi caption masing-masing dan satu caption utama untuk keseluruhan gambar.

Kode berikut menunjukkan cara menyusun tiga gambar secara berdampingan dengan satu caption utama.

\begin{lstlisting}[language=TeX, caption=Kode untuk Menyisipkan Gambar dalam Dokumen dengan Subfigure, label=lst:kode_gambar_multi]
    \begin{figure}
        \centering
        \begin{subfigure}[b]{0.3\textwidth}
            \centering
            \includegraphics[width=\linewidth]{gambar-kucing.jpg}
            \caption{Kucing Lucu 1}
            \label{fig:kucing-a}
        \end{subfigure}
        \hfill
        \begin{subfigure}[b]{0.3\textwidth}
            \centering
            \includegraphics[width=\linewidth]{logo-uny.png}
            \caption{Logo UNY}
            \label{fig:logo-uny-b}
        \end{subfigure}
        \hfill
        \begin{subfigure}[b]{0.3\textwidth}
            \centering
            \includegraphics[width=\linewidth]{gambar-kucing.jpg}
            \caption{Kucing Lucu 2}
            \label{fig:kucing-c}
        \end{subfigure}
        \caption{Beberapa gambar yang disusun menjadi 1 bagian dengan penomoran (a), (b), dan (c)}
        \label{fig:kucingdanUNY}
    \end{figure}
\end{lstlisting}

\noindent Berikut adalah penjelasan dari setiap bagian kode di atas:

\begin{packed_enum}
    \item \texttt{\textbackslash begin\{figure\} ... \textbackslash end\{figure\}}: Lingkungan \texttt{figure} utama yang berfungsi sebagai wadah untuk menyisipkan beberapa gambar dalam satu bagian.
    
    \item \texttt{\textbackslash begin\{subfigure\}[b]\{0.3\textbackslash textwidth\} ... \textbackslash end\{subfigure\}}: Lingkungan \texttt{subfigure} digunakan untuk setiap gambar yang ingin disusun dalam satu bagian. Parameter \texttt{0.3\textbackslash textwidth} mengatur lebar setiap gambar menjadi sepertiga dari lebar teks, sehingga tiga gambar dapat ditampilkan berdampingan dalam satu baris.
    
    \item \texttt{\textbackslash includegraphics[width=\textbackslash linewidth]\{gambar-nama\}}: Memasukkan setiap gambar dengan lebar yang sesuai dengan lebar yang telah ditentukan untuk \texttt{subfigure}. 
        \begin{packed_enum}
            \item Gambar pertama menggunakan file \texttt{gambar-kucing}, dengan caption "Kucing Lucu 1".
            \item Gambar kedua menggunakan file \texttt{logo-uny}, dengan caption "Logo UNY".
            \item Gambar ketiga juga menggunakan file \texttt{gambar-kucing}, dengan caption "Kucing Lucu 2".
        \end{packed_enum}
    
    \item \texttt{\textbackslash hfill}: Menyisipkan ruang kosong antar gambar, agar setiap \texttt{subfigure} memiliki jarak yang merata.
    
    \item \texttt{\textbackslash caption\{...\}}: Caption utama yang menjelaskan ketiga gambar sekaligus. Caption ini akan ditampilkan di bawah semua gambar dalam lingkungan \texttt{figure}.

    \item \texttt{\textbackslash label\{fig:kucingdanUNY\}}: Memberikan label untuk keseluruhan kelompok gambar, sehingga kita bisa merujuk ke seluruh bagian gambar ini dalam teks dengan \texttt{\textbackslash cref\{fig:kucingdanUNY\}}.
\end{packed_enum}

Dengan menggunakan metode ini, Anda dapat menyisipkan beberapa gambar dalam satu bagian dengan satu caption utama seperti pada \cref{fig:kucingdanUNY}. Setiap gambar dapat memiliki caption terpisah dan nomor (misalnya, (a), (b), (c)), sehingga rujukan spesifik untuk masing-masing gambar dapat dibuat, seperti \texttt{\textbackslash cref\{fig:kucing-a\}} untuk merujuk ke \cref{fig:kucing-a}.

\begin{figure}
    \centering
    \begin{subfigure}[b]{0.3\textwidth}
        \centering
        \includegraphics[width=\linewidth]{gambar-kucing.jpg}
        \caption{Kucing Lucu 1}
        \label{fig:kucing-a}
    \end{subfigure}
    \hfill
    \begin{subfigure}[b]{0.3\textwidth}
        \centering
        \includegraphics[width=\linewidth]{logo-uny.png}
        \caption{Logo UNY}
        \label{fig:logo-uny-b}
    \end{subfigure}
    \hfill
    \begin{subfigure}[b]{0.3\textwidth}
        \centering
        \includegraphics[width=\linewidth]{gambar-kucing.jpg}
        \caption{Kucing Lucu 2}
        \label{fig:kucing-c}
    \end{subfigure}
    \caption{Beberapa gambar yang disusun menjadi 1 bagian dengan penomoran (a), (b), dan (c)}
    \label{fig:kucingdanUNY}
\end{figure}

\subsection{Membuat Tabel}
Pada bagian ini akan dijelaskan bagaimana membuat tabel dalam sebuah dokumen \LaTeX. untuk membuat tabel memang agak sedikit sulit, sehingga saya menyarankan menggunakan tool berikut \url{https://www.tablesgenerator.com/} atau \url{https://www.latex-tables.com/} kemudian isikan tabel pada tool generator tersebut dan salin kodenya ke dalam dokumen \LaTeX. Berikut adalah contoh dari sebuah tabel yang telah dibuat. Jangan lupa setiap tabel harus dimention dan dijelaskan dibacaan seperti berikut ini \cref{tab:hresult}. Contoh pembuatan tabel terlihat kodenya pada \cref{lst:kode_tabel}.

\begin{lstlisting}[language=TeX, caption=Kode untuk Membuat Tabel dalam Dokumen, label=lst:kode_tabel]
    \begin{table}[h]
        \caption{Performance Using Hard Decision Detection}
        \label{tab:hresult}
        \centering
        \begin{tabular}{c rrrrrrr}
            \hline\hline
            Audio Name&\multicolumn{7}{c}{Sum of Extracted Bits} \\ [0.5ex] 
            \hline
            Police & 5 & -1 & 5& 5& -7& -5& 3\\
            Midnight & 7 & -3 & 5& 3& -1& -3& 5\\
            News & 9 & -3 & 7& 9& -5& -1& 9\\[0.8ex]
            \hline
        \end{tabular}
    \end{table}
\end{lstlisting}

\noindent Berikut adalah penjelasan dari setiap bagian kode di atas:

\begin{packed_enum}
    \item \texttt{\textbackslash begin\{table\}[h] ... \textbackslash end\{table\}}: Lingkungan \texttt{table} digunakan untuk membuat tabel dan menempatkannya di posisi tertentu dalam dokumen. Parameter \texttt{[h]} menginstruksikan LaTeX untuk menempatkan tabel di posisi yang paling mendekati lokasi kode tersebut dalam teks. Jika posisi ini tidak berfungsi dengan baik, Anda bisa menggunakan parameter lain, seperti \texttt{[H]} (dari paket \texttt{float}) untuk menempatkan tabel di lokasi yang lebih spesifik.
    \item \texttt{\textbackslash caption\{Performance Using Hard Decision Detection\}}: Menambahkan keterangan (caption) di atas tabel. Caption ini akan otomatis ditampilkan dalam Daftar Tabel dan diberi nomor secara otomatis oleh LaTeX.
    \item \texttt{\textbackslash label\{tab:hresult\}}: Memberi label pada tabel, memungkinkan tabel dirujuk dalam teks menggunakan perintah \texttt{\textbackslash cref\{tab:hresult\}} atau \texttt{\textbackslash ref\{tab:hresult\}}, yang akan menghasilkan "Tabel 1" atau sesuai penomoran tabel.
    \item \texttt{\textbackslash centering}: Mengatur tabel agar berada di tengah halaman.
    \item \texttt{\textbackslash begin\{tabular\}\{c rrrrrrr\} ... \textbackslash end\{tabular\}}: Lingkungan \texttt{tabular} digunakan untuk membuat struktur tabel. Pengaturan kolom menggunakan karakter:
        \begin{packed_enum}
            \item \texttt{c}: Mengatur kolom pertama rata tengah.
            \item \texttt{r}: Mengatur tujuh kolom berikutnya rata kanan.
        \end{packed_enum}
    \item \texttt{\textbackslash hline}: Menambahkan garis horizontal di tabel. Dua \texttt{\textbackslash hline} berturut-turut digunakan untuk garis ganda pada bagian header tabel.
    \item \texttt{\textbackslash multicolumn\{7\}\{c\}\{Sum of Extracted Bits\}}: Menggabungkan tujuh kolom berikutnya menjadi satu sel besar yang berisi teks "Sum of Extracted Bits", yang disejajarkan ke tengah dengan pengaturan \texttt{c}.
    \item Isi tabel, seperti:
        \begin{packed_enum}
            \item \texttt{Police}: Data pada baris ini terkait audio "Police", dengan tujuh angka di kolom berikutnya yang merepresentasikan "Sum of Extracted Bits".
            \item Baris lain mengikuti format yang sama.
        \end{packed_enum}
    \item Jarak tambahan antara baris terakhir dan \texttt{\textbackslash hline} berikutnya diberikan dengan parameter opsional \texttt{[0.8ex]}, yang menambahkan spasi vertikal untuk keterbacaan.
\end{packed_enum}

Dengan penjelasan ini, kode menghasilkan tabel terstruktur yang diberi nomor secara otomatis dan dapat dirujuk di teks dokumen. Hasil tabel dari \cref{lst:kode_tabel} adalah terlihat pada \cref{tab:hresult}.

\begin{table}[h]
    \caption{Performance Using Hard Decision Detection}
    \label{tab:hresult}
    \centering
    \begin{tabular}{c rrrrrrr}
        \hline\hline
        Audio Name&\multicolumn{7}{c}{Sum of Extracted Bits} \\ [0.5ex] 
        \hline
        Police & 5 & -1 & 5& 5& -7& -5& 3\\
        Midnight & 7 & -3 & 5& 3& -1& -3& 5\\
        News & 9 & -3 & 7& 9& -5& -1& 9\\[0.8ex]
        \hline
    \end{tabular}
\end{table}

Kita juga bisa menambahkan tabel yang besar dengan format halaman landscape seperti contoh berikut dan mention tabel seperti berikut ini \cref{tab:LPer} dan berikut ini \cref{tab:PPer}.

\begin{lstlisting}[language=TeX, caption=Kode untuk Membuat Tabel dalam Dokumen dengan Sideway, label=lst:kode_tabel_sideway]
    \begin{sidewaystable}[htbp]
        \caption{Performance After Post Filtering}
        \label{tab:LPer}
        \centering
        \begin{tabular}{l c c rrrrrrr}
            \hline\hline
            Audio &Audibility & Decision &\multicolumn{7}{c}{Sum of Extracted Bits} 
            \\ [0.5ex] 
            \hline
            & &soft &1 & $-1$ & 1 & 1 & $-1$ & $-1$ & 1 \\[-1ex]
            \raisebox{1.5ex}{Police} & \raisebox{1.5ex}{5}&hard
            & 2 & $-4$ & 4 & 4 & $-2$ & $-4$ & 4 \\[1ex]
            & &soft & 1 & $-1$ & 1 & 1 & $-1$ & $-1$ & 1 \\[-1ex]
            \raisebox{1.5ex}{Beethoven} & \raisebox{1.5ex}{5}& hard
            &8 & $-8$ & 2 & 8 & $-8$ & $-8$ & 6 \\[1ex]
            & &soft & 1 & $-1$ & 1 & 1 & $-1$ & $-1$ & 1 \\[-1ex]
            \raisebox{1.5ex}{Metallica} & \raisebox{1.5ex}{5}& hard
            &4 & $-8$ & 8 & 4 & $-8$ & $-8$ & 8 \\[1ex]
            \hline
        \end{tabular}
    \end{sidewaystable}
\end{lstlisting}

\noindent Berikut adalah penjelasan dari setiap bagian kode di atas:

\begin{packed_enum}
    \item \texttt{\textbackslash begin\{sidewaystable\}[htbp] ... \textbackslash end\{sidewaystable\}}: Lingkungan \texttt{sidewaystable} dari paket \texttt{rotating} digunakan untuk menampilkan tabel dalam orientasi landscape. Parameter \texttt{[htbp]} menunjukkan preferensi posisi tabel pada dokumen. Pastikan Anda telah memuat paket \texttt{rotating} di preamble dengan perintah \texttt{\textbackslash usepackage\{rotating\}}.
    \item \texttt{\textbackslash caption\{Performance After Post Filtering\}}: Menambahkan caption (keterangan) di atas tabel. Caption ini akan otomatis dimasukkan dalam Daftar Tabel dan diberi nomor secara otomatis.
    \item \texttt{\textbackslash label\{tab:LPer\}}: Memberi label pada tabel, memungkinkan Anda merujuk tabel ini dalam teks menggunakan perintah \texttt{\textbackslash cref\{tab:LPer\}} atau \texttt{\textbackslash ref\{tab:LPer\}}, yang akan menghasilkan "Tabel 1" atau sesuai penomoran tabel.
    \item \texttt{\textbackslash centering}: Mengatur tabel agar berada di tengah halaman.
    \item \texttt{\textbackslash begin\{tabular\}\{l c c rrrrrrr\} ... \textbackslash end\{tabular\}}: Lingkungan \texttt{tabular} digunakan untuk membuat struktur tabel dengan pengaturan kolom sebagai berikut:
        \begin{packed_enum}
            \item \texttt{l}: Mengatur kolom pertama rata kiri untuk kolom "Audio".
            \item \texttt{c}: Mengatur kolom kedua dan ketiga rata tengah untuk kolom "Audibility" dan "Decision".
            \item \texttt{r}: Tujuh kolom berikutnya rata kanan untuk data "Sum of Extracted Bits".
        \end{packed_enum}
    \item \texttt{\textbackslash hline}: Menambahkan garis horizontal di tabel. Dua \texttt{\textbackslash hline} berturut-turut digunakan untuk garis ganda pada bagian header tabel.
    \item \texttt{\textbackslash multicolumn\{7\}\{c\}\{Sum of Extracted Bits\}}: Menggabungkan tujuh kolom berikutnya menjadi satu sel besar yang berisi teks "Sum of Extracted Bits", yang disejajarkan ke tengah dengan pengaturan \texttt{c}.
    \item Isi tabel, misalnya:
        \begin{packed_enum}
            \item Data pada baris pertama terkait audio "Police", dengan kolom audibility berisi nilai 5, dan data decision dengan dua opsi: "soft" dan "hard".
            \item Data "soft" pada baris pertama dan "hard" pada baris kedua diisi dengan angka sesuai kolom masing-masing.
            \item Untuk beberapa entri seperti "Police", "Beethoven", dan "Metallica", kolom audibility dan audio di tengah (seperti nilai 5) diangkat dengan perintah \texttt{\textbackslash raisebox} untuk memberikan efek centering pada teks.
        \end{packed_enum}
    \item \texttt{[1ex]} atau \texttt{[-1ex]}: Mengatur jarak antar baris untuk menjaga keterbacaan dan posisi elemen tabel yang lebih seimbang.
\end{packed_enum}

Kode ini akan menghasilkan tabel landscape dengan satu caption, beberapa kolom gabungan, dan penomoran otomatis dan hasilnya terlihat pada \cref{tab:LPer}.

\begin{sidewaystable}[htbp]
    \caption{Performance After Post Filtering}
    \label{tab:LPer}
    \centering
    \begin{tabular}{l c c rrrrrrr}
        \hline\hline
        Audio &Audibility & Decision &\multicolumn{7}{c}{Sum of Extracted Bits} 
        \\ [0.5ex] 
        \hline
        & &soft &1 & $-1$ & 1 & 1 & $-1$ & $-1$ & 1 \\[-1ex]
        \raisebox{1.5ex}{Police} & \raisebox{1.5ex}{5}&hard
        & 2 & $-4$ & 4 & 4 & $-2$ & $-4$ & 4 \\[1ex]
        & &soft & 1 & $-1$ & 1 & 1 & $-1$ & $-1$ & 1 \\[-1ex]
        \raisebox{1.5ex}{Beethoven} & \raisebox{1.5ex}{5}& hard
        &8 & $-8$ & 2 & 8 & $-8$ & $-8$ & 6 \\[1ex]
        & &soft & 1 & $-1$ & 1 & 1 & $-1$ & $-1$ & 1 \\[-1ex]
        \raisebox{1.5ex}{Metallica} & \raisebox{1.5ex}{5}& hard
        &4 & $-8$ & 8 & 4 & $-8$ & $-8$ & 8 \\[1ex]
        \hline
    \end{tabular}
\end{sidewaystable}

Contoh lain \cref{lst:kode_tabel_lain} untuk pembuatan tabel seperti di bawah ini dan hasilnya tertampil pada \cref{tab:PPer}.

\begin{lstlisting}[language=TeX, caption=Kode untuk Membuat Tabel dalam Dokumen, label=lst:kode_tabel_lain]
    \begin{table}[ht]
        \caption{Performance After Post Filtering}
        \label{tab:PPer}
        \centering
        \begin{tabular}{l c c rrrrrrr}
            \hline\hline
            Audio &Audibility & Decision &\multicolumn{7}{c}{Sum of Extracted Bits} 
            \\ [0.5ex] 
            \hline
            & &soft &1 & $-1$ & 1 & 1 & $-1$ & $-1$ & 1 \\[-1ex]
            \raisebox{1.5ex}{Police} & \raisebox{1.5ex}{5}&hard
            & 2 & $-4$ & 4 & 4 & $-2$ & $-4$ & 4 \\[1ex]
            & &soft & 1 & $-1$ & 1 & 1 & $-1$ & $-1$ & 1 \\[-1ex]
            \raisebox{1.5ex}{Beethoven} & \raisebox{1.5ex}{5}& hard
            &8 & $-8$ & 2 & 8 & $-8$ & $-8$ & 6 \\[1ex]
            & &soft & 1 & $-1$ & 1 & 1 & $-1$ & $-1$ & 1 \\[-1ex]
            \raisebox{1.5ex}{Metallica} & \raisebox{1.5ex}{5}& hard
            &4 & $-8$ & 8 & 4 & $-8$ & $-8$ & 8 \\[1ex]
            \hline
        \end{tabular}
    \end{table}
\end{lstlisting}

\begin{table}[ht]
	\caption{Performance After Post Filtering}
	\label{tab:PPer}
	\centering
	\begin{tabular}{l c c rrrrrrr}
		\hline\hline
		Audio &Audibility & Decision &\multicolumn{7}{c}{Sum of Extracted Bits} 
		\\ [0.5ex] 
		\hline
		& &soft &1 & $-1$ & 1 & 1 & $-1$ & $-1$ & 1 \\[-1ex]
		\raisebox{1.5ex}{Police} & \raisebox{1.5ex}{5}&hard
		& 2 & $-4$ & 4 & 4 & $-2$ & $-4$ & 4 \\[1ex]
		& &soft & 1 & $-1$ & 1 & 1 & $-1$ & $-1$ & 1 \\[-1ex]
		\raisebox{1.5ex}{Beethoven} & \raisebox{1.5ex}{5}& hard
		&8 & $-8$ & 2 & 8 & $-8$ & $-8$ & 6 \\[1ex]
		& &soft & 1 & $-1$ & 1 & 1 & $-1$ & $-1$ & 1 \\[-1ex]
		\raisebox{1.5ex}{Metallica} & \raisebox{1.5ex}{5}& hard
		&4 & $-8$ & 8 & 4 & $-8$ & $-8$ & 8 \\[1ex]
		\hline
	\end{tabular}
\end{table}

\subsection{Menambahkan Persamaan}

Persamaan tidak lepas dari bidang ilmu teknik dan kadang perlu dituliskan dalam sebuah laporan. Sangat mudah menuliskan persamaan pada sebuah dokumen \LaTeX. Terdapat 2 jenis penulisan persamaan, yaitu inline dengan text seperti contoh ini \(x^2 + y^2 = z^2\) atau seperti ini $E=mc^2$. Jenis lain adalah dituliskan seperti di bawah ini, yang otomatis akan mendapatkan penomoran. Apabila belum familiar dengan kode untuk penulisan persamaan pada \LaTeX, Anda bisa menggunakan tool berikut \url{https://latex.codecogs.com/eqneditor/editor.php} atau \url{https://latexeditor.lagrida.com/}. Setiap persamaan harus disebutkan dalam teks seperti \cref{eq:satu} dan \cref{eq:equationDua} dan dijelaskan terkait persamaan tersebut untuk apa.

\begin{lstlisting}[language=TeX, caption=Kode untuk Menulis Persamaan, label=lst:kode_persamaan_emc]
    \begin{equation}
        \label{eq:satu}
        E=mc^2
    \end{equation}
\end{lstlisting}

\begin{lstlisting}[language=TeX, caption=Kode untuk Menulis Persamaan, label=lst:kode_persamaan_mn]
    \begin{equation}
        \label{eq:equationDua}
        m_n = k_p*e_n + \frac{k_e*T}{T_{reset}}\sum_{i=0}^{n}e_i + k_d\frac{e_n - e_{n-1}}{\delta t} + m_{R}
    \end{equation}
\end{lstlisting}

\noindent Berikut adalah penjelasan dari setiap bagian kode di atas:

Dengan menggunakan lingkungan \texttt{equation}, Anda bisa menulis dan memberi nomor persamaan secara otomatis serta merujuknya dengan mudah dalam teks menggunakan \texttt{\textbackslash cref}.

\begin{equation}
    \label{eq:satu}
    E=mc^2
\end{equation}

\begin{equation}
    \label{eq:equationDua}
    m_n = k_p*e_n + \frac{k_e*T}{T_{reset}}\sum_{i=0}^{n}e_i + k_d\frac{e_n - e_{n-1}}{\delta t} + m_{R}
\end{equation}

\subsection{Referensi dan Sitasi}
Referensi dan sitasi pada dokumen \LaTeX juga cukup mudah. Silahkan buka file \textit{pustaka.bib} dan amati beberapa contoh penulisan referensi yang ada. Untuk menggenerate bentuk referensi seperti ini dapat menggunakan Mendeley atau Zotero. Mensitasi referensi seperti ini \citep{Priambodo_2021}, \citep{Nasuha_2017}, \citep{Dhewa_Dharmawan_Priyambodo_2017}, \citep{Arifin_2015} dapat dilakukan dengan perintah \verb|\citep{nama_label}|. Pemberian sitasi dengan benar membuat sitasi tersebut dapat di klik dan akan mengarahkan ke daftar pustaka.