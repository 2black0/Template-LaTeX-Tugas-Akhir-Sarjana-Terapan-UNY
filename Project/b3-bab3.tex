%==================================================================
% Ini adalah bab 3
% Silahkan edit sesuai kebutuhan, baik menambah atau mengurangi \section, \subsection
%==================================================================

\chapter[KONSEP RANCANGAN ALAT DAN PENGUJIAN]{\\ KONSEP RANCANGAN ALAT DAN PENGUJIAN}

\section{Metode Pengerjaan Project Berbasis Engineering Design Process}
Bagian ini menjelaskan metode pengerjaan proyek yang mengadopsi pendekatan Engineering Design Process. Pendekatan ini digunakan untuk memastikan perancangan dan pengembangan proyek dilakukan secara sistematis, mulai dari identifikasi masalah hingga evaluasi akhir. Langkah-langkah dalam metode ini membantu dalam mencapai solusi yang optimal dan terukur.

\subsection{Identifikasi Masalah}
Tahap identifikasi masalah bertujuan untuk menguraikan permasalahan utama yang dihadapi dan memerlukan solusi. Pada tahap ini, dilakukan analisis untuk memahami aspek-aspek penting dari masalah dan menentukan faktor-faktor yang perlu diatasi melalui proyek ini.

\subsection{Definisi Kebutuhan}
Berdasarkan masalah yang telah diidentifikasi, tahap ini berfokus pada definisi kebutuhan proyek secara jelas dan terstruktur. Kebutuhan ini meliputi spesifikasi teknis, fungsi yang diinginkan, serta kriteria-kriteria lain yang harus dipenuhi agar solusi dapat berfungsi dengan baik.

\subsection{Generasi Ide dan Solusi}
Pada tahap ini, berbagai ide dan solusi alternatif dikembangkan dan dievaluasi. Bagian ini menjelaskan proses brainstorming untuk menghasilkan ide yang inovatif, termasuk analisis terhadap kelebihan dan kekurangan dari setiap alternatif solusi yang diusulkan.

\subsection{Perencanaan dan Desain Awal}
Desain awal sistem dikembangkan berdasarkan solusi yang dipilih, dengan mempertimbangkan aspek teknis dan kebutuhan yang telah didefinisikan. Bagian ini menyajikan perencanaan mengenai struktur sistem, alat, dan bahan yang akan digunakan, serta jadwal kerja proyek secara keseluruhan.

\subsection{Pembuatan Prototipe}
Prototipe dibuat untuk menguji konsep dan desain awal dari sistem. Bagian ini menguraikan langkah-langkah dalam proses pembuatan prototipe, termasuk alat dan bahan yang diperlukan, serta tantangan yang mungkin dihadapi selama proses.

\subsection{Pengujian dan Evaluasi}
Prototipe yang telah dibuat kemudian diuji untuk menilai kinerjanya terhadap kebutuhan dan spesifikasi yang telah ditetapkan. Bagian ini menjelaskan prosedur pengujian yang diterapkan, metode pengumpulan data, serta analisis terhadap hasil yang diperoleh.

\subsection{Perbaikan dan Penyempurnaan}
Berdasarkan hasil pengujian dan evaluasi, dilakukan perbaikan untuk menyempurnakan sistem. Bagian ini menjelaskan penyesuaian yang dilakukan untuk meningkatkan kinerja sistem, serta proses iterasi yang dilakukan hingga mencapai hasil yang optimal.

\section{Perancangan Sistem Elektronika}
Bagian ini menguraikan perancangan dari sisi elektronika yang menjadi inti dari sistem. Perancangan ini mencakup blok diagram, pemilihan komponen, dan perancangan rangkaian.

\subsection{Blok Diagram Sistem}
Blok diagram sistem memberikan gambaran umum mengenai arsitektur sistem secara keseluruhan. Bagian ini menyajikan diagram beserta penjelasan fungsi setiap blok yang terdapat di dalam sistem, termasuk bagaimana setiap blok berinteraksi.

\subsection{Pemilihan dan Spesifikasi Komponen Elektronika}
Pemilihan komponen elektronika dilakukan berdasarkan kebutuhan dari desain sistem. Bagian ini menjelaskan spesifikasi teknis dari setiap komponen yang digunakan, seperti mikrokontroler, sensor, aktuator, dan komponen pendukung lainnya.

\subsection{Perancangan Rangkaian Elektronika}
Perancangan rangkaian elektronika bertujuan untuk mencapai fungsionalitas yang diinginkan dari sistem. Bagian ini menguraikan skema rangkaian, penjelasan aliran arus dan tegangan, serta hubungan antar komponen dalam sistem.

\section{Perancangan Mekanik}
Bagian ini membahas perancangan mekanik dari sistem yang mendukung komponen elektronik secara fisik, termasuk struktur dan pemilihan bahan.

\subsection{Spesifikasi Desain Mekanik}
Desain mekanik disusun berdasarkan kebutuhan fisik sistem untuk memastikan bahwa komponen elektronika terlindungi dan dapat berfungsi dengan baik. Bagian ini menjelaskan spesifikasi teknis desain mekanik yang digunakan.

\subsection{Pemilihan Bahan dan Komponen Mekanik}
Pemilihan bahan didasarkan pada kriteria seperti kekuatan, ketahanan, dan biaya. Bagian ini menguraikan bahan dan komponen mekanik yang digunakan untuk konstruksi sistem, serta alasan pemilihan bahan tersebut.

\subsection{Desain Struktur dan Konstruksi}
Struktur dan konstruksi sistem dirancang untuk memberikan dukungan fisik yang stabil. Bagian ini menjelaskan proses desain dan konstruksi dari struktur mekanik sistem, serta tantangan yang mungkin dihadapi.

\section{Perancangan Perangkat Lunak}
Bagian ini mencakup perancangan perangkat lunak yang mengendalikan sistem, meliputi diagram alir, pengembangan kode, dan pengujian perangkat lunak.

\subsection{Flowchart atau Diagram Alir Perangkat Lunak}
Diagram alir menggambarkan alur kerja dari perangkat lunak yang mengontrol sistem. Bagian ini menyajikan flowchart lengkap yang menunjukkan logika dan struktur kontrol dari perangkat lunak.

\subsection{Pemrograman dan Pengembangan Kode}
Kode perangkat lunak dikembangkan untuk mendukung fungsionalitas sistem. Bagian ini menguraikan struktur dan logika dari program yang dibuat, bahasa pemrograman yang digunakan, serta strategi pengembangan kode.

\subsection{Pengujian Kode Perangkat Lunak}
Setelah kode perangkat lunak dikembangkan, dilakukan pengujian untuk memastikan bahwa perangkat lunak berfungsi sesuai yang diharapkan. Bagian ini menjelaskan metode pengujian, jenis uji (misalnya, uji unit dan uji integrasi), serta hasil yang diperoleh.

\section{Perancangan Integrasi Sistem}
Bagian ini menjelaskan proses integrasi antara komponen elektronik, mekanik, dan perangkat lunak agar sistem dapat bekerja sebagai satu kesatuan.

\subsection{Integrasi Komponen Elektronika, Mekanik, dan Perangkat Lunak}
Proses integrasi bertujuan untuk menyatukan seluruh komponen sistem agar berfungsi sebagai satu kesatuan. Bagian ini menguraikan langkah-langkah dalam proses integrasi dan memastikan semua komponen bekerja secara sinkron.

\subsection{Pengujian Awal dan Penyempurnaan Integrasi}
Pengujian awal dilakukan setelah integrasi untuk menilai performa sistem secara keseluruhan. Bagian ini menjelaskan hasil pengujian integrasi dan modifikasi yang diperlukan untuk meningkatkan kinerja sistem.

\section{Rencana Pengujian}
Bagian ini merencanakan pengujian yang komprehensif terhadap sistem untuk memastikan bahwa semua komponen dan fungsi bekerja dengan baik.

\subsection{Metode Pengujian Sistem}
Metode pengujian dipilih berdasarkan tujuan dan kebutuhan proyek. Bagian ini menjelaskan berbagai metodologi pengujian yang dirancang, seperti uji kinerja, uji stabilitas, dan uji kompatibilitas.

\subsection{Prosedur Pengujian}
Prosedur pengujian disusun untuk menguji sistem secara menyeluruh, mulai dari pengaturan awal hingga pelaksanaan uji. Bagian ini menjelaskan langkah-langkah pengujian secara detail untuk menjamin konsistensi dan keandalan hasil.

\subsection{Kriteria Keberhasilan Pengujian}
Kriteria keberhasilan ditentukan untuk mengevaluasi kinerja sistem berdasarkan parameter-parameter tertentu. Bagian ini menguraikan kriteria-kriteria keberhasilan yang digunakan, seperti ketepatan, keandalan, dan efisiensi sistem.

Pada bagian ini akan dijelaskan beberapa tahap utama dalam proses pengembangan proyek dan beberapa rekomendasi peningkatan yang dapat dilakukan. Daftar tahapan pengembangan proyek akan disajikan dalam bentuk bernomor untuk menunjukkan urutan logis dari proses, sementara rekomendasi peningkatan akan disajikan dalam bentuk daftar berpoin untuk mempermudah identifikasi setiap item secara mandiri.