%==================================================================
% Ini adalah bab 2
% Silahkan edit sesuai kebutuhan, baik menambah atau mengurangi \section, \subsection
%==================================================================

\chapter[TINJAUAN PUSTAKA]{\\ TINJAUAN PUSTAKA}

\section{Teori Dasar Komponen Elektronika}
Bagian ini membahas teori dasar mengenai komponen elektronika yang digunakan dalam perancangan sistem. Setiap komponen memiliki karakteristik khusus yang mempengaruhi kinerja keseluruhan sistem. Pemahaman tentang karakteristik komponen-komponen ini sangat penting dalam mengoptimalkan fungsi dan stabilitas rangkaian.

\subsection{Jenis dan Karakteristik Komponen Pasif}
Komponen pasif seperti resistor, kapasitor, dan induktor memiliki fungsi dasar dalam pengaturan arus dan tegangan dalam sirkuit. Bagian ini menguraikan jenis-jenis komponen pasif serta karakteristik utama yang mempengaruhi performa dan fungsi komponen tersebut dalam sirkuit elektronika.

\subsection{Jenis dan Karakteristik Komponen Aktif}
Komponen aktif, seperti transistor, dioda, dan IC, memainkan peran penting dalam penguatan dan pengaturan sinyal. Bagian ini menjelaskan berbagai jenis komponen aktif yang digunakan dalam proyek serta karakteristik utamanya, yang menentukan efektivitas dan efisiensi sistem elektronika.

\subsection{Peran dan Fungsi Modul dalam Sistem Elektronika}
Modul-modul elektronika memberikan fungsionalitas tambahan yang membantu dalam memperkuat performa sistem. Bagian ini mengulas modul-modul yang sering digunakan, seperti modul daya atau komunikasi, serta peran masing-masing dalam mendukung integrasi sistem yang lebih efisien.

\subsection{Analisis Daya dan Efisiensi Komponen}
Analisis daya dan efisiensi komponen adalah aspek penting dalam desain sistem yang hemat energi. Bagian ini membahas cara-cara mengevaluasi dan mengoptimalkan daya yang dikonsumsi oleh komponen, yang berperan dalam meningkatkan efisiensi energi dari sistem secara keseluruhan.

\section{Sistem dan Teknik Rangkaian Elektronika}
Bagian ini membahas berbagai sistem dan teknik yang digunakan dalam perancangan rangkaian elektronika, baik analog maupun digital. Setiap teknik ini memungkinkan sistem berfungsi dengan lebih efektif sesuai dengan kebutuhan aplikasi.

\subsection{Konsep Dasar Rangkaian Analog}
Rangkaian analog digunakan untuk memproses sinyal kontinu dan memainkan peran penting dalam berbagai aplikasi. Bagian ini menjelaskan prinsip-prinsip dasar yang digunakan dalam rangkaian analog, termasuk elemen-elemen utamanya dan penggunaannya.

\subsection{Konsep Dasar Rangkaian Digital}
Rangkaian digital beroperasi dengan sinyal diskrit, cocok untuk pemrosesan informasi digital. Bagian ini menguraikan prinsip dasar rangkaian digital serta komponen-komponen utama yang mendukung fungsi-fungsi digital dalam proyek ini.

\subsection{Teknik Pengolahan Sinyal pada Sistem Elektronika}
Pengolahan sinyal adalah proses penting untuk interpretasi informasi dari lingkungan. Bagian ini membahas metode umum dalam pengolahan sinyal yang diterapkan pada sistem elektronika, termasuk teknik yang digunakan dalam pemfilteran atau pemrosesan data.

\subsection{Pengkabelan dan Pengaturan Sirkuit untuk Keandalan Sistem}
Pengkabelan dan tata letak yang baik meningkatkan keandalan sistem secara keseluruhan. Bagian ini menguraikan teknik pengkabelan dan pengaturan sirkuit yang efektif, serta bagaimana hal ini dapat mempengaruhi performa sistem.

\subsection{Pengendalian dan Penggerak (Motor Driver, Relay, dsb.)}
Bagian ini menjelaskan penggunaan penggerak seperti motor driver dan relay untuk menggerakkan komponen mekanis. Diperlukan teknik pengendalian khusus untuk memastikan bahwa setiap penggerak bekerja sesuai dengan tujuan sistem.

\section{Teknologi yang Digunakan}
Bagian ini mengulas teknologi yang umum digunakan dalam proyek berbasis elektronika, seperti mikrokontroler, sensor, aktor, dan teknologi komunikasi nirkabel. Teknologi ini memungkinkan sistem untuk merespons lingkungan dan berinteraksi dengan pengguna.

\subsection{Mikrokontroler dan Mikroprosesor}
Mikrokontroler dan mikroprosesor berfungsi sebagai unit pemrosesan utama dalam sistem elektronika. Bagian ini menguraikan arsitektur dasar, bahasa pemrograman yang relevan, serta protokol komunikasi yang sering digunakan dalam proyek ini.

\subsection{Sensor dan Aktuator}
Sensor dan aktuator memungkinkan interaksi sistem dengan lingkungannya. Bagian ini membahas jenis-jenis sensor yang digunakan, cara kerja, dan integrasinya ke dalam sistem agar sistem dapat mengumpulkan data dan merespons secara aktif.

\subsection{Teknologi Nirkabel}
Teknologi nirkabel seperti Bluetooth dan Wi-Fi memungkinkan komunikasi jarak jauh dalam sistem IoT. Bagian ini menguraikan jenis teknologi nirkabel yang relevan, termasuk protokol komunikasi dan aspek keamanan yang perlu dipertimbangkan.

\section{Metode Kontrol dan Kecerdasan Buatan}
Bagian ini membahas metode kontrol dan kecerdasan buatan yang diterapkan dalam sistem elektronika untuk mencapai pengendalian yang lebih cerdas dan otomatis, seperti kontrol PID, logika fuzzy, dan deep learning.

\subsection{Pengendalian PID (Proportional-Integral-Derivative)}
PID adalah metode kontrol yang efektif dalam mengatur respons sistem. Bagian ini menjelaskan prinsip dasar PID, aplikasinya dalam pengaturan sistem elektronika, serta teknik tuning yang dapat meningkatkan stabilitas dan respons sistem.

\subsection{Fuzzy Logic Control}
Logika fuzzy adalah metode kontrol fleksibel yang sering digunakan dalam sistem nonlinear. Bagian ini menjelaskan konsep dasar logika fuzzy, serta cara implementasi dan manfaatnya dalam pengendalian sistem yang kompleks.

\subsection{Deep Learning}
Deep learning memungkinkan sistem untuk belajar dari data, yang sangat berguna dalam aplikasi otomatisasi. Bagian ini menguraikan algoritma dasar dalam deep learning, seperti CNN dan RNN, serta penerapannya dalam pengembangan sistem IoT.

\subsection{Perbandingan dan Pemilihan Metode yang Sesuai}
Bagian ini membahas perbandingan antara berbagai metode kontrol yang tersedia, menjelaskan kelebihan dan kekurangannya masing-masing, serta bagaimana memilih metode yang paling sesuai untuk aplikasi proyek ini.

\section{Konsep Engineering Design Process}
Engineering Design Process adalah metodologi sistematis yang digunakan untuk merancang sistem secara efektif. Bagian ini menguraikan prinsip utama dan langkah-langkah dari Engineering Design Process dalam konteks pengembangan sistem elektronika.

\subsection{Pengertian dan Langkah-langkah Engineering Design Process}
Engineering Design Process adalah proses desain iteratif yang mencakup beberapa tahapan untuk mencapai desain yang optimal. Bagian ini menjelaskan langkah-langkah dasar yang terlibat dan bagaimana proses ini diadaptasi dalam proyek ini.

\subsection{Aplikasi Engineering Design Process pada Proyek Elektronika}
Bagian ini membahas penerapan Engineering Design Process dalam proyek elektronika untuk mencapai hasil desain yang optimal. Diuraikan langkah-langkah praktis dalam menerapkan metodologi ini.

\subsection{Studi Kasus Implementasi Engineering Design Process dalam Desain Elektronika}
Studi kasus ini menunjukkan contoh penerapan Engineering Design Process dalam desain sistem yang relevan dengan proyek. Dengan studi kasus ini, pembaca dapat memahami implementasi proses desain secara nyata.

\subsection{Teknik Evaluasi dan Optimasi Desain}
Teknik evaluasi dan optimasi desain merupakan langkah penting dalam proses desain yang berkelanjutan. Bagian ini menjelaskan metode yang digunakan untuk mengevaluasi dan menyempurnakan desain agar mencapai hasil terbaik.

\section{Penelitian Terdahulu yang Relevan}
Bagian ini berisi ulasan terhadap penelitian terdahulu yang relevan dengan proyek ini. Tujuannya adalah untuk melihat pendekatan yang telah digunakan, menemukan kelebihan dan kekurangannya, serta mengidentifikasi inovasi yang dapat dikembangkan.

\subsection{Tinjauan Penelitian Terdahulu tentang Proyek Serupa}
Bagian ini membahas penelitian terdahulu yang serupa dengan proyek ini. Tinjauan ini bertujuan untuk memahami bagaimana proyek ini dapat memberikan kontribusi yang berbeda atau lebih baik.

\subsection{Analisis Kekurangan dan Kelebihan Metode pada Penelitian Terdahulu}
Setiap metode yang digunakan dalam penelitian terdahulu memiliki kelebihan dan kekurangan. Bagian ini mengidentifikasi aspek yang perlu diperbaiki atau dikembangkan lebih lanjut berdasarkan analisis metode-metode tersebut.

\subsection{Inovasi dan Kontribusi yang Dibawa dalam Penelitian Ini}
Penelitian ini membawa inovasi tertentu yang berkontribusi dalam memperkaya hasil penelitian sebelumnya. Bagian ini menjelaskan kontribusi utama proyek ini terhadap bidang elektronika, serta perbedaan yang ditawarkan.