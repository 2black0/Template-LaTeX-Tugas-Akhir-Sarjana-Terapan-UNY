%==================================================================
% Ini adalah bab 4
% Silahkan edit sesuai kebutuhan, baik menambah atau mengurangi \section, \subsection
%==================================================================

\chapter[HASIL DAN PEMBAHASAN]{\\ HASIL DAN PEMBAHASAN}

\section{Hasil Implementasi Sistem}
Bagian ini menyajikan hasil dari implementasi sistem, termasuk pengujian terhadap komponen individu dan keseluruhan sistem untuk memastikan bahwa sistem berfungsi sesuai dengan desain yang telah direncanakan. Setiap hasil pengujian yang diperoleh akan menjadi dasar dalam mengevaluasi performa akhir dari sistem.

\subsection{Pengujian Komponen Individu}
Pengujian komponen individu dilakukan untuk memastikan bahwa setiap bagian sistem berfungsi sesuai spesifikasi sebelum diintegrasikan ke dalam sistem utama. Bagian ini memuat hasil pengujian komponen seperti sensor, aktuator, dan modul komunikasi, serta evaluasi terhadap kinerjanya berdasarkan standar yang telah ditetapkan.

\subsection{Pengujian Keseluruhan Sistem}
Setelah semua komponen individu dinyatakan berfungsi dengan baik, pengujian dilakukan pada sistem yang telah terintegrasi. Bagian ini menguraikan hasil pengujian untuk memastikan bahwa semua komponen bekerja secara sinergis dan sistem dapat memenuhi spesifikasi fungsional yang telah direncanakan.

\section{Analisis Kinerja Sistem}
Analisis kinerja sistem dilakukan untuk mengevaluasi efektivitas dan efisiensi sistem dalam mencapai tujuan yang telah ditetapkan. Bagian ini berfokus pada pengukuran dan analisis terhadap parameter-parameter utama yang mencerminkan kualitas kinerja sistem.

\subsection{Evaluasi Terhadap Spesifikasi Desain}
Evaluasi dilakukan untuk menilai sejauh mana sistem yang dikembangkan memenuhi spesifikasi desain awal. Bagian ini menyajikan perbandingan antara hasil pengujian dengan spesifikasi teknis dan fungsional yang telah ditetapkan, serta mengidentifikasi bagian-bagian yang sesuai atau tidak sesuai.

\subsection{Analisis Parameter Kinerja Utama}
Parameter kinerja utama, seperti kecepatan, akurasi, dan efisiensi, dianalisis untuk menilai performa sistem secara komprehensif. Bagian ini menyajikan hasil pengukuran terhadap parameter-parameter tersebut, membandingkannya dengan target yang diinginkan, dan memberikan interpretasi mengenai kinerja sistem.

\subsection{Perbandingan dengan Sistem yang Sudah Ada}
Untuk mengetahui keunggulan dan kekurangan sistem yang dikembangkan, dilakukan perbandingan kinerja dengan sistem serupa yang sudah ada atau hasil penelitian terdahulu. Bagian ini menganalisis perbedaan dalam hal performa, efisiensi, dan fungsi, serta memberikan penjelasan mengenai kelebihan dan kekurangan sistem yang diusulkan.

\section{Kendala dan Solusi yang Ditemukan Selama Pengujian}
Selama proses pengujian, berbagai kendala teknis dapat muncul yang berpengaruh terhadap hasil akhir sistem. Bagian ini mengidentifikasi kendala yang ditemukan, solusi yang diterapkan untuk mengatasinya, dan penyempurnaan sistem berdasarkan hasil evaluasi.

\subsection{Kendala Teknis dalam Pengujian}
Bagian ini menguraikan kendala teknis yang muncul selama proses pengujian, seperti masalah pada komponen atau gangguan komunikasi antar perangkat. Dampak dari kendala ini terhadap kinerja sistem juga dibahas untuk memberikan gambaran tentang tantangan yang dihadapi.

\subsection{Solusi Terhadap Kendala}
Untuk mengatasi kendala yang ditemukan, solusi atau tindakan tertentu diterapkan. Bagian ini menjelaskan solusi yang diterapkan pada setiap kendala, seperti modifikasi komponen, pengaturan ulang perangkat, atau penyesuaian parameter.

\subsection{Penyempurnaan Berdasarkan Hasil Pengujian}
Berdasarkan hasil pengujian dan evaluasi kendala, penye
