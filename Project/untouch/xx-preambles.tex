%==================================================================
% Konfigurasi utama untuk dokumen LaTeX
%==================================================================

%% DILARANG EDIT BAGIAN INI

% Mengatur bahasa dokumen ke Bahasa Indonesia dan encoding karakter
\usepackage[indonesian]{babel}
\usepackage[utf8]{inputenc}

% Pengaturan jarak antar baris dan penyesuaian kotak teks agar rata kiri
\usepackage{setspace}
\usepackage[raggedrightboxes]{ragged2e}

% Paket untuk memasukkan gambar dalam dokumen
\usepackage{graphicx}
\graphicspath{{gambar/}} % Menentukan folder default untuk gambar
\usepackage{float}       % Mengatur posisi gambar dalam teks

% Mengatur indentasi paragraf
\usepackage{indentfirst}  % Memastikan paragraf pertama di setiap section memiliki indentasi
\setlength\parindent{1cm} % Mengatur jarak indentasi paragraf menjadi 1 cm

% Mengatur font utama dokumen menjadi Times New Roman
\usepackage{pslatex}

% Mengatur format penomoran section dan subsection
\renewcommand{\thesection}{\Alph{section}.\hspace{0.18cm}} % Section: A, B, C, ...
\renewcommand{\thesubsection}{\arabic{subsection}.}         % Subsection: 1, 2, 3, ...

% Pengaturan daftar isi, daftar gambar, dan daftar tabel
\usepackage{tocloft}
\cftsetpnumwidth{1.5em}
\cftsetrmarg{2em}
\setlength{\cftsecnumwidth}{1.5em}
\setlength{\cftsubsecnumwidth}{1.5em}
\renewcommand{\cftchapdotsep}{\cftdotsep}
\setlength{\cftbeforechapskip}{3pt}

% Penyesuaian judul bab dalam daftar isi agar ditampilkan sebagai "BAB"
\renewcommand\cftchappresnum{BAB }
\renewcommand\cftchapaftersnum{}
\newlength\mylen
\settowidth\mylen{\bfseries BAB 1 :\ } % Menyesuaikan lebar penomoran bab
\cftsetindents{chap}{0pt}{\mylen}

% Pengaturan format dan penomoran judul bab
\usepackage{titlesec}
\titleformat{\chapter}{\doublespacing\fontsize{14pt}{16pt}\bfseries}{\MakeUppercase{\chaptertitlename\ \Roman{chapter}}\filcenter}{0.15cm}{\centering\uppercase}
\titleformat{\section}{\fontsize{12}{14}\bfseries}{\thesection}{0.4cm}{}
\titleformat{\subsection}{\fontsize{12}{14}\bfseries}{\thesubsection}{0.675cm}{}

% Mengatur jarak dan format spacing untuk chapter, section dan subsection
\titlespacing*{\chapter}{0pt}{-1cm}{20pt}
\titlespacing*{\section}{0pt}{10pt}{0cm}
\titlespacing*{\subsection}{0pt}{10pt}{0cm}

% Pengaturan untuk caption gambar dan tabel
\usepackage[font=small, format=plain, labelfont=bf, up, textfont=up]{caption}
\usepackage{subcaption}

% Menghapus tanda titik dua pada caption
\captionsetup[figure]{labelsep=space}
\captionsetup[table]{labelsep=space}

% Mengatur nomor caption gambar dan tabel sesuai bab
\renewcommand{\thefigure}{\arabic{chapter}.\arabic{figure}}
\renewcommand{\thetable}{\arabic{chapter}.\arabic{table}}

% Mengatur hyphenation (pemisahan kata) agar lebih rapi
\tolerance=1
\emergencystretch=\maxdimen
\hyphenpenalty=10000
\hbadness=10000

% Pengaturan tabel, multirow, dan ukuran kolom otomatis
\usepackage{tabularx}
\usepackage{multirow}

% Pengaturan margin halaman
\usepackage{geometry}
\geometry{
    left=4cm,          % Margin kiri
    top=3cm,           % Margin atas
    right=3cm,         % Margin kanan
    bottom=3cm,        % Margin bawah
}

% Paket untuk notasi matematika
\usepackage{amsmath}

% Pengaturan nomor pada persamaan matematika sesuai bab
%\renewcommand{\theequation}{\arabic{chapter}.\arabic{equation}}
\makeatletter
\renewcommand{\theequation}{\arabic{chapter}.\arabic{equation}}
\renewcommand{\@eqnnum}{\theequation}
\def\tagform@#1{\maketag@@@{#1}} % This line removes the parentheses
\makeatother

% Untuk halaman berorientasi landscape
\usepackage{rotating}

% Pengaturan untuk list (daftar item dan angka)
\usepackage{enumitem}
\setlist{nosep} % Menghilangkan jarak antar item dalam list
\newenvironment{packed_enum}{ % Membuat lingkungan untuk daftar bernomor
    \begin{enumerate}[leftmargin=1.5\parindent]
        \setlength{\itemsep}{0pt}
        \setlength{\parskip}{0pt}
        \setlength{\parsep}{0pt}
        }{\end{enumerate}}

\newenvironment{packed_item}{ % Membuat lingkungan untuk daftar berpoin
    \begin{itemize}[leftmargin=1.375\parindent]
        \setlength{\itemsep}{0pt}
        \setlength{\parskip}{0pt}
        \setlength{\parsep}{0pt}
        }{\end{itemize}}

% Paket untuk bibliografi menggunakan BibTeX
\usepackage[sort]{natbib}
\bibliographystyle{apalike} % Ganti dengan apalike jika menggunakan Overleaf

% Paket untuk tabel yang panjang dan melampaui satu halaman
\usepackage{longtable}

% Paket untuk menampilkan kode program - HARUS DIMUAT SEBELUM HYPERREF
\usepackage{xcolor}
\usepackage{listings}

% Paket untuk memasukkan hyperlink dalam dokumen - HARUS DIMUAT SETELAH LISTINGS
\usepackage{hyperref}

%\AtBeginDocument{\renewcommand{\lstlistingname}{Kode}} 
%\AtBeginDocument{\renewcommand{\thelstlisting}{\thechapter.\arabic{lstlisting}}}
%\AtBeginDocument{\renewcommand{\thelstlisting}{\arabic{chapter}.\arabic{lstlisting}.}}
\AtBeginDocument{\renewcommand{\lstlistingname}{Kode}}
\AtBeginDocument{\renewcommand{\thelstlisting}{\arabic{chapter}.\arabic{lstlisting}}}

\captionsetup[lstlisting]{
  format=plain,
  labelfont=bf,
  justification=centering,
  singlelinecheck=false,
  labelsep=space
}
\definecolor{codegreen}{rgb}{0,0.6,0}
\definecolor{codegray}{rgb}{0.5,0.5,0.5}
\definecolor{codepurple}{rgb}{0.58,0,0.82}
\definecolor{backcolour}{rgb}{0.95,0.95,0.92}
\lstdefinestyle{mystyle}{
    backgroundcolor=\color{backcolour},   
    commentstyle=\color{codegreen},
    keywordstyle=\color{magenta},
    numberstyle=\tiny\color{codegray},
    stringstyle=\color{codepurple},
    basicstyle=\ttfamily\footnotesize,
    columns=fullflexible,
    breakatwhitespace=false,         
    breaklines=true,                 
    captionpos=b,                    
    keepspaces=true,                 
    numbers=left,                    
    numbersep=5pt,                  
    showspaces=false,                
    showstringspaces=false,
    showtabs=false,                  
    tabsize=2,
    lineskip=-1pt
}
\lstset{style=mystyle}

%paket untuk gambar dengan tikz
\usepackage{tikz}
\usepackage{pgfplots}
\usepackage{pgf-pie}
\pgfplotsset{compat=1.18}
\usetikzlibrary{shapes.geometric, arrows, positioning, calc, shapes}
\tikzstyle{startstop} = [rectangle, rounded corners, minimum width=3cm, minimum height=1cm,text centered, draw=black, fill=red!30]
\tikzstyle{process} = [rectangle, minimum width=3cm, minimum height=1cm, text centered, draw=black, fill=orange!30]
\tikzstyle{decision} = [diamond, minimum width=3cm, minimum height=1cm, text centered, draw=black, fill=green!30]
\tikzstyle{arrow} = [thick,->,>=stealth]

%untuk if-then
\usepackage{ifthen}

% PRACTICAL SOLUTION FOR CLEVEREF LABEL ISSUE
% Due to a complex bug interaction between packages, we'll use a different approach
% Remove cleveref temporarily and use standard referencing with custom formatting

% For now, comment out cleveref and use standard LaTeX references
% Users can uncomment the line below if they want to try cleveref again
% \usepackage[nameinlink,capitalise]{cleveref}

% Custom reference commands that provide similar functionality to cleveref
\newcommand{\figref}[1]{gambar~\ref{#1}}
\newcommand{\Figref}[1]{Gambar~\ref{#1}}
\newcommand{\tabref}[1]{tabel~\ref{#1}}
\newcommand{\Tabref}[1]{Tabel~\ref{#1}}
\newcommand{\equref}[1]{persamaan~(\ref{#1})}
\newcommand{\Equref}[1]{Persamaan~(\ref{#1})}
\newcommand{\lstref}[1]{kode~\ref{#1}}
\newcommand{\Lstref}[1]{Kode~\ref{#1}}

% If cleveref is needed, users should replace \cref{fig:xxx} with \figref{fig:xxx} etc.

%% DILARANG EDIT BAGIAN INI