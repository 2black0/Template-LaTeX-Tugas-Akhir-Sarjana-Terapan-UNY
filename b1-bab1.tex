%==================================================================
% Ini adalah bab 1
% Silahkan edit sesuai kebutuhan, baik menambah atau mengurangi \section, \subsection
%==================================================================

\chapter[PENDAHULUAN]{\\ PENDAHULUAN}

\section{Latar Belakang}
Bagian ini menjelaskan alasan dan motivasi di balik pengembangan proyek ini. Proyek ini diinisiasi untuk mengatasi kebutuhan atau permasalahan tertentu yang timbul dalam bidang teknologi, industri, atau aplikasi sehari-hari. Melalui identifikasi kebutuhan ini, diperoleh wawasan tentang bagaimana proyek dapat memberikan solusi yang efektif dan efisien. Latar belakang ini juga menguraikan pentingnya topik proyek dalam perkembangan teknologi terkini, serta bagaimana penerapan teknologi atau metode yang diusulkan dapat membawa nilai tambah. Selain itu, bagian ini memberikan gambaran singkat mengenai situasi atau tren teknologi saat ini yang mempengaruhi pengembangan proyek.

\section{Rumusan Masalah}
Rumusan masalah adalah langkah penting yang merangkum permasalahan spesifik atau kebutuhan yang menjadi dasar pengembangan proyek. Pada bagian ini, dijelaskan permasalahan utama yang dihadapi, seperti keterbatasan pada sistem atau perangkat yang sudah ada, atau kebutuhan baru yang belum terpenuhi oleh teknologi saat ini. Identifikasi masalah juga mencakup tantangan teknis, fungsional, atau ekonomi yang menjadi penghambat dan bagaimana proyek ini diharapkan dapat menjawab permasalahan tersebut. Fokusnya adalah memberikan pemahaman yang jelas mengenai alasan pentingnya mengembangkan proyek ini sebagai solusi yang dibutuhkan.

\section{Tujuan Proyek}
Tujuan proyek menyatakan secara spesifik hasil atau capaian yang diinginkan dari pengembangan proyek ini. Bagian ini dirancang untuk memastikan bahwa proyek memiliki sasaran yang jelas dan terukur. Tujuan tersebut dirumuskan berdasarkan masalah yang telah diidentifikasi dan mencakup pencapaian tertentu, seperti peningkatan kinerja sistem, efisiensi, atau kemudahan penggunaan yang diharapkan. Selain itu, tujuan proyek dapat berupa pengembangan prototipe, penerapan teknologi tertentu, atau pencapaian fungsionalitas baru yang belum ada. Penjabaran tujuan yang jelas membantu menjaga fokus proyek dan memberikan arah yang tepat dalam setiap tahapan pengembangan.

\section{Manfaat Proyek}
Manfaat proyek menguraikan dampak positif yang diharapkan dari hasil proyek ini bagi pengguna, industri, atau masyarakat secara umum. Manfaat ini mencakup berbagai aspek, seperti kontribusi terhadap peningkatan produktivitas, pengurangan biaya, peningkatan kualitas, atau kemudahan dalam penggunaan teknologi. Selain manfaat langsung, proyek ini juga diharapkan memiliki dampak jangka panjang yang bermanfaat, seperti mendorong inovasi di bidang terkait atau membuka peluang baru untuk pengembangan lebih lanjut. Dengan menjelaskan manfaat proyek, pembaca dapat memahami nilai tambah yang dihadirkan oleh proyek ini.

\section{Batasan Proyek}
Batasan proyek mengidentifikasi cakupan dan batasan ruang lingkup pengembangan sistem atau perangkat yang dirancang. Bagian ini mencakup aspek-aspek yang akan menjadi fokus utama dalam pengembangan serta aspek yang akan dikecualikan dari lingkup proyek. Penjelasan batasan ini penting agar proyek tetap terarah dan tidak meluas ke aspek-aspek yang berada di luar tujuan awal. Batasan proyek juga mencakup keterbatasan teknis, waktu, atau sumber daya yang mempengaruhi desain dan implementasi sistem. Dengan menetapkan batasan, proyek ini dapat lebih terfokus dan efisien dalam pencapaiannya.

\section{Keaslian Gagasan}
Keaslian gagasan bertujuan untuk menekankan inovasi atau kontribusi unik yang ditawarkan oleh proyek ini. Bagian ini menjelaskan bagaimana proyek ini menawarkan pendekatan yang berbeda atau peningkatan dibandingkan dengan metode atau perangkat yang sudah ada. Keaslian gagasan dapat diperlihatkan melalui perbandingan dengan proyek atau produk serupa, menunjukkan perbedaan signifikan atau keunggulan yang dihadirkan oleh solusi yang diusulkan. Misalnya, peningkatan kinerja, efisiensi, atau kemudahan penggunaan yang dihasilkan dari metode atau pendekatan baru. Selain itu, bagian ini juga bisa mencakup penggunaan teknologi atau desain yang belum banyak diterapkan dalam konteks yang sama. Penekanan pada keaslian gagasan membantu menunjukkan bahwa proyek ini tidak hanya mengikuti pola yang sudah ada, tetapi juga menghadirkan sesuatu yang baru dan relevan.

\section{Sistematika Penulisan}
Sistematika penulisan memberikan panduan mengenai struktur dari keseluruhan laporan proyek ini, sehingga memudahkan pembaca dalam memahami alur isi laporan dari setiap bab. Bagian ini menjelaskan isi dari setiap bab secara singkat, mulai dari latar belakang hingga kesimpulan dan rekomendasi. Misalnya, BAB I membahas pendahuluan dan dasar pengembangan proyek, BAB II menguraikan tinjauan pustaka dan landasan teori, dan seterusnya. Dengan memberikan sistematika penulisan, pembaca dapat memahami bagaimana laporan ini disusun secara keseluruhan dan bagaimana setiap bab saling berkaitan dalam mencapai tujuan akhir proyek.